We aren't
\chapter*{Afterword}
\markboth{Afterword}{}
\label{chapter:afterword}


It is hard, you will agree, for any considering human to avoid forming a mental picture of the way that quantitative techniques will influence commerce, and society, going forward. Let's face it, {\em all future visions} of {\em just about anything} place mathematically driven automation (by various names) in a central role. 

(Even the fictional President Josiah Bartlett, of {\em The West Wing}, told us that the 21st Century would be the century of statistics.) 

Evidently my own vision includes a new type of utility. The task of catalyzing a prediction web is better addressed by code, not prose, and I really must get back to that. But I hope you take some time to consider the potential for competitive prediction in the ``small''. 

As with computing prophesies of yesteryear that only saw enterprise use, I suspect we are mostly blind to what the  equivalent of an in-house data science team will eventually be used for, once shrunk down to the size of a thimble. I'm sure I'll look back on these pages in a few years and kick myself for missing something obvious. 

I'm reminded too of how difficult it is to overcome our mental inertia, especially on the matters of cost - the central obsession of this book. Many years ago my undergraduate professor, filling out a form to acquire me a university computer account, listed ``email'' as the only justification. I was shocked at the time as computers were expensive. It seemed unlikely that an administrator would allocate precious computation and bandwidth to someone whose only stated use was so frivolous. 
 
Needless to say I got that computer account, and how quaint that guilt seems now. In the same way, I think we will use a network of micro-predicting algorithms and data in ways that initially seem fatuous. Later, the exact same uses will be considered essential. The idea of not being able to map the near future, of everything large and small, will seem as antiquated as a car that cannot see the road ahead. 

Admittedly, we won't all use microprediction at the lowest level of implementation detail, but that's okay. We aren't all experts in the TCP/IP internet protocols, either. Whatever your prefered level of abstraction is, I hope you are interested in helping to create an incredibly inexpensive alternative to the ``data science project''. A universal source of intelligence, despite the limitations I've stated, could be the next great utility. 

To emphasize one last time, I have concerned myself with the microprediction domain only, for reasons that I hope are now clear. It is not {\em the} future of AI. It is not {\em the} future of statistics. It is not the future of general artificial intelligence - merely half the things branded AI. As I write these words, the world is experiencing a dreadful pandemic. It needs thoughtful inferential statistics to interpret a dire situation, make medium term forecasts, and approve vaccines. 

This kind of statistics will never be replaced by a microprediction network. But a substrate where algorithms travel can help in surprising ways - as with the sourcing of surrogate models for disease spread, or crowd-sourced approximations for long-running molecular simulations.  

And {\em in addition} to medium term prediction, or the things we usually associate with the word prediction, the world also needs inexpensive, accessible but accurate microprediction to drive operational efficiency in every industry. This is well within our collective ability. The prediction web is ambitious in some ways, but rather mundane in others. It almost feels like busywork. 

Recently I came across a paper whose subtitle summed up the ambition nicely:
\begin{quote}{\cite{stacking}}
    All models are wrong, but some are somewhere useful.
\end{quote}
Perhaps you have such a model, but aren't quite sure of all the places it might add value. 

I've suggested that given the right infrastructure, algorithms that manage algorithms can collectively solve this problem for you, and that they will be empowered by all the results from game theory, contest theory, auction theory, sequential experimental design and anything else that assists. They don't need humans. 

Or perhaps you are one of many who drive the data science contest participation rates well beyond the Nash equilibrium. I can offer you a challenge much closer to real-world work. 

For those of you who pick this book up hoping for some new algorithmic breakthrough, I'm sorry I've disappointed. But we can do a lot with what we have. Algorithms merely need access to problems. They need businesses to hold their hands up and say ``{\em sure, I'll give this explicit microprediction thing a shot}". Maybe you can do that. 

Like models, all visions of the future are wrong, but maybe some are (somewhere) useful. I hope that's true of mine. My main goal has been to provoke you into contemplation of the fundamental contradiction at the heart of machine learning - or at least its artisan production. The efficacy of data hungry methods counters the very notion that humans should be hand-managing them. 

I thank you for your suspension of disbelief, and hope to interact with you, or your autonomous brain-children, at a future date. 
